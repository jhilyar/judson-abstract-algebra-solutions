\chapter{Preliminaries}\label{ch:preliminaries}

\section{}\label{sec:1}

Suppose that
\begin{align*}
A &= \{ x : x \in \mathbb{N} \text{ and } x \text{ is even}\}, \\
B &= \{ x : x \in \mathbb{N} \text{ and } x \text{ is prime}\}, \\
C &= \{ x : x \in \mathbb{N} \text{ and } x \text{ is a multiple of 5}\}.
\end{align*}

Describe each of the following sets.
\hr
\begin{exlist}
    \item $A \cap B = \{ 2 \}.$
    \item $B \cap C = \{ 5 \}.$
    \item $A \cup B = \{ x : x \in \mathbb{N} \text{ and } x \text{ is even or } x \text{ is prime}\}.$
    \item $A \cap (B \cup C) = \{ x : x \in \mathbb{N} \text{ and } x = 2 \text{ or } x \text{ is a multiple of 10}\}.$
\end{exlist}

\section{}\label{sec:2}

If $A = \{ a, b, c \}$, $B = \{ 1, 2, 3 \}$, $C = \{ x \}$, and $D = \emptyset$, list all of the elements in each of the following sets.
\hr
\begin{exlist}
    \item $A \times B = \{ (a,1), (a,2), (a,3), (b,1), (b,2), (b,3), (c,1), (c,2), (c,3)\}.$
    \item $B \times C = \{ (1,x), (2,x), (3,x)\}.$
    \item $A \times B \times C = \{ (a,1,x), (a,2,x), (a,3,x), (b,1,x), (b,2,x), (b,3,x), (c,1,x), (c,2,x), (c,3,x)\}.$
    \item $A \times D = \emptyset.$
\end{exlist}
\pagebreak
\section{}\label{sec:3}

Find an example of two nonempty sets $A$ and $B$ for which $A \times B = B \times A$.
\hr

For any nonempty set $S$, let $A = B = S$.

\section{}\label{sec:4}

Prove $A \cup \emptyset = A$ and $A \cap \emptyset = \emptyset$.
\hr
\begin{align*}
    A \cup \emptyset &= \{x : x \in A \lor x \in \emptyset\} \\
    &= \{x : x \in A\} \\
    &= A \\
    \\
    A \cap \emptyset &= \{x : x \in A \land x \in \emptyset\} \\
    &= \{\} \\
    &= \emptyset
\end{align*}\qed

\section{}\label{sec:5}

Prove $A \cup B = B \cup A$ and $A \cap B = B \cap A$.
\hr
\begin{align*}
    A \cup B &= \{x : x \in A \lor x \in B\} \\
    &= \{x : x \in B \lor x \in A\} \\
    &= B \cup A \\
    \\
    A \cap B &= \{x : x \in A \land x \in B\} \\
    &= \{x : x \in B \land x \in A\} \\
    &= B \cap A
\end{align*}\qed

\section{}\label{sec:6}

Prove $A \cup (B \cap C) = (A \cup B) \cap (A \cup C)$.
\hr
\begin{align*}
    A \cup (B \cap C) &= \{x : x \in A \lor x \in B \cap C\} \\
    &= \{x : x \in A \lor (x \in B \land x \in C)\} \\
    &= \{x : (x \in A \lor x \in B) \land (x \in A \lor x \in C)\} \\
    &= \{x : x \in (A \cup B) \lor x \in (A \cup C)\} \\
    &= \{x : x \in (A \cup B) \cap (A \cup C)\} \\
    &= (A \cup B) \cap (A \cup C)
\end{align*}\qed
\pagebreak
\section{}\label{sec:7}

Prove $A \cap (B \cup C) = (A \cap B) \cup (A \cap C)$.
\hr
\begin{align*}
    A \cap (B \cup C) &= \{x : x \in A \land x \in B \cup C\} \\
    &= \{x : x \in A \land (x \in B \lor x \in C)\} \\
    &= \{x : (x \in A \land x \in B) \lor (x \in A \land x \in C)\} \\
    &= \{x : x \in (A \cap B) \land x \in (A \cap C)\} \\
    &= \{x : x \in (A \cap B) \cup (A \cap C)\} \\
    &= (A \cap B) \cup (A \cap C)
\end{align*}\qed

\section{}\label{sec:8}

Prove $A \subset B$ if and only if $A \cap B = A$.

\hr
By definition
\[A \subset B \iff (x \in A \implies x \in B).\]

We know that $x \in B$ implies $x \in A \cap B$, and so by transitivity of implication,
\[A \subset B \iff (x \in A \implies x \in A \cap B),\]

or in other words,
\[A \subset B \iff A \subset A \cap B.\]

As we trivially have that $A \cap B \subset A$, therefore
\[A \subset B \iff A \cap B = A.\]\qed

\section{}\label{sec:9}

Prove $(A \cap B)' = A' \cup B'$.
\hr
\begin{align*}
    (A \cap B)' &= \{x : x \notin A \cap B\} \\
    &= \{x : x \notin A \lor x \notin B\} \\
    &= A' \cup B'
\end{align*}\qed
\pagebreak
\section{}\label{sec:10}

Prove $A \cup B = (A \cap B) \cup (A \setminus B) \cup (B \setminus A)$.
\hr

If $x \in A \cup B$, then either $x$ is in both $A$ and $B$, $x$ is only in $A$, or $x$ is only in $B$.

\medskip

Therefore
\begin{align*}
    A \cup B &= \{x : x \in A \cup B\} \\
    &= \{x : x \in A \cap B \lor x \in A \setminus B \lor x \in B \setminus A\} \\
    &= (A \cap B) \cup (A \setminus B) \cup (B \setminus A).
\end{align*}\qed

\section{}\label{sec:11}

Prove $(A \cup B) \times C = (A \times C) \cup (B \times C)$.
\hr
\begin{align*}
    (A \cup B) \times C &= \{(x, y) : x \in A \cup B \land y \in C\} \\
    &= \{(x, y) : (x \in A \lor x \in B) \land y \in C\} \\
    &= \{(x, y) : (x \in A \land y \in C) \lor (x \in B \land y \in C)\} \\
    &= (A \times C) \cup (B \times C)
\end{align*}\qed

\section{}\label{sec:12}

Prove $(A \cap B) \setminus B = \emptyset$.
\hr
\begin{align*}
    (A \cap B) \setminus B &= \{x : x \in (A \cap B) \land x \notin B\} \\
    &= \{x : x \in A \land x \in B \land x \notin B\} \\
    &= \{\} \\
    &= \emptyset
\end{align*}\qed

\section{}\label{sec:13}

Prove $(A \cup B) \setminus B = A \setminus B$.
\hr
\begin{align*}
    (A \cup B) \setminus B &= \{x : x \in (A \cup B) \land x \notin B\} \\
    &= \{x : (x \in A \lor x \in B) \land x \notin B\} \\
    &= \{x : (x \in A \land x \notin B) \lor (x \in B \land x \notin B)\} \\
    &= \{x : x \in A \land x \notin B\} \\
    &= A \setminus B
\end{align*}\qed
\pagebreak
\section{}\label{sec:14}

Prove $A \setminus (B \cup C) = (A \setminus B) \cap (A \setminus C)$.
\hr
\begin{align*}
    A \setminus (B \cup C) &= \{x : x \in A \land x \notin (B \cup C)\} \\
    &= \{x : x \in A \land (x \notin B \land x \notin C)\} \\
    &= \{x : (x \in A \land x \notin B) \land (x \in A \land x \notin C)\} \\
    &= (A \setminus B) \cap (A \setminus C)
\end{align*}\qed

\section{}\label{sec:15}

Prove $A \cap (B \setminus C) = (A \cap B) \setminus (A \cap C)$.
\hr
\begin{align*}
    (A \cap B) \setminus (A \cap C) &= \{x : x \in A \cap B \land x \notin A \cap C\} \\
    &= \{x : (x \in A \land x \in B) \land (x \notin A \land x \notin C)\} \\
    &= \{x : (x \in A \land x \in B \land x \notin A) \land (x \in A \land x \in B \land x \notin C)\} \\
    &= \{x : x \in A \land x \in B \land x \notin C\} \\
    &= \{x : x \in A \land x \in B \setminus C\} \\
    &= A \cap (B \setminus C)
\end{align*}\qed

\section{}\label{sec:16}

Prove $(A \setminus B) \cup (B \setminus A) = (A \cup B) \setminus (A \cap B)$.
\hr
\begin{align*}
    (A \setminus B) \cup (B \setminus A) &= \{x : x \in (A \setminus B) \lor x \in (B \setminus A)\} \\
    &= \{x : (x \in A \land x \notin B) \lor (x \in B \land x \notin A)\} \\
    &= \{x : (x \in A \lor x \in B) \land (x \notin B \lor x \in B) \\
    & \qquad \qquad \land (x \in A \lor x \notin A) \land (x \notin B \lor x \notin A)\} \\
    &= \{x : (x \in A \lor x \in B) \land (x \notin B \lor x \notin A)\} \\
    &= \{x : (x \in A \lor x \in B) \land (x \in B \land x \in A)'\} \\
    &= \{x : x \in A \cup B \land x \notin A \cap B\} \\
    &= (A \cup B) \setminus (A \cap B) \\
\end{align*}\qed
\pagebreak
\section{}\label{sec:17}

Which of the following relations $f:\mathbb{Q} \rightarrow \mathbb{Q}$ define a mapping? In each case, supply a reason why $f$ is or is not a mapping.
\hr
\begin{exlist}
    \item $f(p/q) = \frac{p + 1}{p - 2}$

    No. $\frac{1}{3} = \frac{2}{6}$, but $f(1/3) = 2 \neq \frac{3}{4} = f(2/6)$.

    \item $f(p/q) = \frac{3p}{3q}$

    Yes. For any fraction $\frac{p}{q} \in \mathbb{Q}$, the value of $f(p/q)$ reduces to simply $\frac{p}{q}$, and therefore $f$ is the identity mapping.

    \item $f(p/q) = \frac{p + q}{q^2}$

    No. $\frac{1}{3} = \frac{2}{6}$, but $f(1/3) = \frac{4}{9} \neq \frac{2}{9} = f(2/6)$.

    \item $f(p/q) = \frac{3p^2}{7q^2} - \frac{p}{q}$

    Yes. Let $x = \frac{p}{q}$. Then $f(x) = \frac{3}{7}x^2 - x$, and so it follows that $f(p/q)$ isn't dependent on the representation of $p/q$.
\end{exlist}

\section{}\label{sec:18}

Determine which of the following functions are one-to-one and which are onto. If the function is not onto, determine its range.
\hr
\begin{exlist}
    \item $f : \mathbb{R} \rightarrow \mathbb{R}$ defined by $f(x) = e^x$

    \medskip

    Both one-to-one and onto.

    \item $f : \mathbb{Z} \rightarrow \mathbb{Z}$ defined by $f(n) = n^2 + 3$

    \medskip

    Neither. This function is not one-to-one \emdash $f(n) = f(-n)$ \emdash and the range of this function is $\{3, 4, 7, 12, \dots\} \subset \mathbb{N}$.

    \item $f : \mathbb{R} \rightarrow \mathbb{R}$ defined by $f(x) = \sin x$

    \medskip

    Neither. This function is not one-to-one \emdash $f(x) = f(x + 2\pi)$ \emdash and the range of this function is $[-1, 1]$.

    \item $f : \mathbb{Z} \rightarrow \mathbb{Z}$ defined by $f(n) = n^2$

    \medskip

    Neither. This function is not one-to-one \emdash $f(n) = f(-n)$ \emdash and the range of this function is $\{0, 1, 4, 9, \dots\} \subset \mathbb{N}$.
\end{exlist}
\pagebreak
\section{}\label{sec:19}

Let $f: A \rightarrow B$ and $g: B \rightarrow C$ be invertible mappings; that is, mappings such that $f^{-1}$ and $g^{-1}$ exist.
Show that $(g \circ f)^{-1} = f^{-1} \circ g^{-1}$.
\hr

Take any $a \in A$ and let $b \in B$, $c \in C$ be such that $f(a) = b$ and $g(b) = c$, or equivalently that $(g \circ f)(a) = c$. It thus follows that $(g \circ f)^{-1}(c) = a.$

Now, evaluate $(f^{-1} \circ g^{-1})(c)$. By our previous definitions, we have $f^{-1}(g^{-1}(c)) = f^{-1}(b) = a$.

\qed

\section{}\label{sec:20}

Define a function $f: \mathbb{N} \rightarrow \mathbb{N}$ that is one-to-one but not onto.
\hr
\[f(n) = n + 1\]

For any two $m, n \in \mathbb{N}$, $m \neq n$ implies $m + 1 \neq n + 1$. However, there is no $n \in \mathbb{N}$ such that $f(n) = 0$.

\section{}\label{sec:21}

Prove the relation defined on $\mathbb{R}^2$ by $(x_1, y_1) \sim (x_2, y_2)$ if $x_1^2 + y_1^2 = x_2^2 + y_2^2$ is an equivalence relation.
\hr

Reflexivity: For any element $(x,y) \in \mathbb{R}^2$, $x^2 + y^2 = x^2 + y^2$, and so $(x, y) \sim (x, y)$.

\smallskip

Symmetry: For any two elements $(x_1, y_1), (x_2, y_2) \in \mathbb{R}^2$ such that $(x_1, y_1) \sim (x_2, y_2)$, equivalently
$x_1^2 + y_1^2 = x_2^2 + y_2^2$, then clearly we have $x_2^2 + y_2^2 = x_1^2 + y_1^2$, and so $(x_2, y_2) \sim (x_1, y_1)$.

\smallskip

Transitivity: For any three elements $(x_1, y_1), (x_2, y_2), (x_3, y_3) \in \mathbb{R}^2$ such that $(x_1, y_1) \sim (x_2, y_2)$ and $(x_2, y_2) \sim (x_3, y_3)$, equivalently $x_1^2 + y_!^2 = x_2^2 + y_2^2$ and $x_2^2 + y_2^2 = x_3^2 + y_3^2$, then clearly we have $x_1^2 + y_!^2 = x_3^2 + y_3^2$, and so $(x_1, y_1) \sim (x_3, y_3)$.\qed

\section{}\label{sec:22}

Let $f:A \rightarrow B$ and $g:B \rightarrow C$ be maps.

\begin{exlist}
    \item If $f$ and $g$ are both one-to-one functions, show that $g \circ f$ is one-to-one.
    \hrlist

    For $a_n \in A$, let $b_n \in B$, $c_n \in C$ be such that $f(a_n) = b_n$ and $g(b_n) = c_n)$. As $f$ is a one-to-one function, $a_1 \neq a_2$ implies that $b_1 \neq b_2$. Similarly, as $g$ is a one-to-one function, $b_1 \neq b_2$ implies that $c_1 \neq c_2$. Therefore, $a_1 \neq a_2$ implies $c_1 \neq c_2$, and so $g \circ f$ is also one-to-one.
    \pagebreak
    \item\label{gfo} If $g \circ f$ is onto, show that $g$ is onto.
    \hrlist

    Given that $g \circ f$ is onto, then for every $c_n \in C$ there exists some $a_n \in A$ such that $(g \circ f)(a_n) = c_n$. For every $c_n \in C$, then, it is clear that there exists $b_n \in B = f(a_n)$ such that $g(b_n) = c_n$.
    \item\label{gf11} If $g \circ f$ is one-to-one, show that $f$ is one-to-one.
    \hrlist

    For $a_n \in A$, let $b_n \in B$, $c_n \in C$ be such that $f(a_n) = b_n$ and $g(b_n) = c_n)$. Assume that $f$ is not one-to-one. Then there exists some $a_1, a_2 \in A$ such that $a_1 \neq a_2$ but $b_1 = b_2 = b$. As $g \circ f$ is one-to-one, we know that we must have that $c_1 \neq c_2$. However, this implies that $g(b)$ maps to distinct values $c_1$ and $c_2$, which is impossible as we know that $g$ is a mapping. Therefore, $f$ is one-to-one.
    \item If $g \circ f$ is one-to-one and $f$ is onto, show that $g$ is one-to-one.
    \hrlist

    Let $b_1, b_2 \in B$ such that $b_1 \neq b_2$. Given that $f$ is onto and (by~\ref{gf11}) one-to one, we must have that for every $b_n$, there exists some unique $a_n$ such that $f(a_n) = b_n$, and so we know that $a_1 \neq a_2$ must both also exist in $A$. As $g \circ f$ is one-to-one, we have that $a_1 \neq a_2$ implies $c_1 \neq c_2$. Therefore, $b_1 \neq b_2$ also implies that $c_1 \neq c_2$, and so $g$ must be one-to-one.
    \item If $g \circ f$ is onto and $g$ is one-to-one, show that $f$ is onto.
    \hrlist

    Given that $g$ is one-to-one and (by~\ref{gfo}) onto, we know that for all $b_n \in B$ there exists a unique $c_n \in C$ such that $g(b_n) = c_n$. As $(g \circ f)$ is onto, it follows that for all $c_n \in C$ there exists at least one $a_n$ such that $(g \circ f)(a_n) = c_n$. It then follows directly that for all $b_n \in B$ there exists at least one $a_n$ such that $f(a_n) = b_n$, and so $f$ is onto.
\end{exlist}
\pagebreak
\section{}\label{sec:23}

Define a function on the real numbers by
\[f(x) = \frac{x + 1}{x - 1}.\]

What are the domain and range of $f$? What is the inverse of $f$? Compute $f \circ f^{-1}$ and $f^{-1} \circ f$.
\hr
\begin{alignat*}{5}
    \text{Domain:}  && \mathbb{R} \setminus \{1\} \\
    \text{Range:}  && \mathbb{R} \setminus \{1\} \\
    f^{-1}: &&\,y = \frac{x+1}{x-1} &\iff y(x-1) = x+1 \\
            &&&\iff yx-y = x+1 \\
            &&&\iff yx-x = y+1 \\
            &&&\iff x(y-1) = y+1 \\
            &&&\iff x = \frac{y+1}{y-1} \\
            &&&\iff f^{-1}(x) = f(x) \\
    \\
    f \circ f^{-1}: &&\,(f \circ f^{-1})(x) = &\,\frac{\frac{x+1}{x-1}+1}{\frac{x+1}{x-1}-1} \\
                    &&= &\,\frac{\frac{x+1}{x-1}+\frac{x-1}{x-1}}{\frac{x+1}{x-1}-\frac{x-1}{x-1}} \\
                    &&= &\,\frac{\frac{2x}{x-1}}{\frac{2}{x-1}} \\
                    &&= &\,\frac{2x}{x-1} \cdot \frac{x-1}{2} \\
                    &&= &\,x \\
    f^{-1} \circ f: &&\omit\,\rlap{As $f = f^{-1}$, this is the same result as $f \circ f^{-1}$.}
\end{alignat*}

\section{}\label{sec:24}

Let $f:X \rightarrow Y$ be a map with $A_1, A_2 \subset X$ and $B_1, B_2 \subset Y$.

\begin{exlist}
    \item Prove $f(A_1 \cup A_2) = f(A_1) \cup f(A_2)$.
    \hrlist

    Let $y \in f(A_1 \cup A_2)$. By definition, this means there is some $x \in A_1 \cup A_2$ where $f(x) = y$. It must be the case that $x \in A_1 \lor x \in A_2$, and so $f(x) \in f(A_1) \lor f(x) \in f(A_2)$. Therefore $f(x) \in f(A_1) \cup f(A_2)$, and so $f(A_1 \cup A_2) \subset f(A_1) \cup f(A_2)$.

    \medskip

    The opposite direction follows similarly.

    \qed
    \pagebreak
    \item Prove $f(A_1 \cap A_2) \subset f(A_1) \cap f(A_2)$. Give an example in which equality fails.
    \hrlist

    Let $y \in f(A_1 \cap A_2)$. By definition, this means there is some $x \in A_1 \cap A_2$ where $f(x) = y$. It must be the case that $x \in A_1 \land x \in A_2$, and so $f(x) \in f(A_1) \land f(x) \in f(A_2)$. Therefore $f(x) \in f(A_1) \cap f(A_2)$, and so $f(A_1 \cup A_2) \subset f(A_1) \cup f(A_2)$.

    \medskip

    Now, let $y \in f(A_1) \cap f(A_2)$. Therefore, $y \in f(A_1) \land y \in f(A_2)$. The proof of containment in the opposite direction breaks down here, however: this only establishes that there must exist some $x_1, x_2 \in X$ such that $x_1 \in A_1 \land x_2 \in A_2$, with no guarantee that $x_1 = x_2$ or that either element is in $A_1 \cap A_2$.

    \medskip

    For example, let $f(x) = x^2$ for $f: \mathbb{Z} \rightarrow \mathbb{Z}$, with $A_1 = \mathbb{Z}^+$ and $A_2 = \mathbb{Z}^-$. For any $m \in \mathbb{Z}^+$ and $y = m^2$, it is certainly the case that $y \in f(\mathbb{Z}^+) \cap f(\mathbb{Z}^-)$, as $f(m) = y$ and $f(-m) = y$. However, $\mathbb{Z}^+ \cap \mathbb{Z}^- = \emptyset$, and so $y$ could not possibly be an element of $f(\emptyset) = \emptyset$.

    \medskip

    (Note that if $f$ is one-to-one, then we do know that $f(x_1) = f(x_2)$ implies $x_1 = x_2$, and so we can achieve equality of sets.)
    
    \qed
    \item\label{funcunion} Prove $f^{-1}(B_1 \cup B_2) = f^{-1}(B_1) \cup f^{-1}(B_2)$, where $f^{-1}(B) = \{x \in X : f(x) \in B\}$.
    \hrlist

    Let $x \in f^{-1}(B_1 \cup B_2)$. By definition, this means there is some $y \in B_1 \cup B_2$ where $f^{-1}(y) = x$. It must be the case that $y \in B_1 \lor y \in B_2$, and so $f^{-1}(y) \in f^{-1}(B_1) \lor f^{-1}(y) \in f^{-1}(B_2)$. Therefore $f^{-1}(y) \in f^{-1}(B_1) \cup f^{-1}(B_2)$, and so $f^{-1}(B_1 \cup B_2) \subset f^{-1}(B_1) \cup f^{-1}(B_2)$.

    \medskip

    The opposite direction follows similarly.

    \qed
    \item Prove $f^{-1}(B_1 \cap B_2) = f^{-1}(B_1) \cap f^{-1}(B_2)$.
    \hrlist

    Let $x \in f^{-1}(B_1 \cap B_2)$. By definition, this means there is some $y \in B_1 \cap B_2$ where $f^{-1}(y) = x$. It must be the case that $y \in B_1 \land y \in B_2$, and so $x \in f^{-1}(B_1) \land x \in f^{-1}(B_2)$. Therefore $x \in f^{-1}(B_1) \cap f^{-1}(B_2)$, and so $f^{-1}(B_1 \cap B_2) \subset f^{-1}(B_1) \cap f^{-1}(B_2)$.

    \medskip

    Now, let $x \in f^{-1}(B_1) \cap f^{-1}(B_2)$. Therefore, $x \in f^{-1}(B_1) \land x \in f^{-1}(B_2)$. In contrast to~\ref{funcunion}, we can continue further here: this implies that there must exist some $y_1, y_2 \in B$ such that $y_1 \in B_1 \land y_2 \in B_2$. However, as $f$ is a mapping, we must have that $y_1 = y_2 = y$, and so $y \in B_1 \cap B_2$, and thus $x \in f^{-1}(B_1 \cap B_2)$. Therefore, $f^{-1}(B_1) \cap f^{-1}(B_2) \subset f^{-1}(B_1 \cap B_2)$.

    \medskip

    Thus, $f^{-1}(B_1 \cap B_2) = f^{-1}(B_1) \cap f^{-1}(B_2)$.

    \qed
    \pagebreak
    \item Prove $f^{-1}(Y \setminus B_1) = X \setminus f^{-1}(B_1)$.
    \hrlist

    Let $x \in f^{-1}(Y \setminus B_1)$. By definition, this means there is some $y \in Y \setminus B_1$ where $f^{-1}(y) = x$. Thus $y \in Y \land y \notin B_1$, and so $x \in X \land\,x \notin f^{-1}(B_1)$. Therefore $x \in X \setminus f^{-1}(B_1)$, and so $f^{-1}(Y \setminus B_1) \subset X \setminus f^{-1}(B_1)$.

    \medskip

    The opposite direction follows similarly.

    \qed
\end{exlist}

\section{}\label{sec:25}

Determine whether or not the following relations are equivalence relations on the given set. If the relation is an equivalent relation, describe the partition given by it. If the relation is not an equivalence relation, state why it fails to be one.

\begin{exlist}
    \item $x \sim y$ in $\mathbb{R}$ if $x \geq y$.

    \smallskip

    No. Take $x$ and $y$ such that $x > y$. Then $x \sim y$ but $y \nsim x$.

    \item $m \sim n$ in $\mathbb{Z}$ if $mn > 0$.

    \smallskip

    No. $0 \nsim 0$.

    \item $x \sim y$ in $\mathbb{R}$ if $|x - y| \leq 4$.

    \smallskip

    No. Take $x$, $y$, and $z$ such that $x - y = 4$ and $y - z = 4$. Then $x \sim y$ and $y \sim z$, but $x - z = 8$ and so $x \nsim z$.

    \item $m \sim n$ in $\mathbb{Z}$ if $m \equiv n \text{ (mod 6)}$.

    \smallskip

    Yes. All properties follow directly from the definition of a modulus class.
\end{exlist}

\section{}\label{sec:26}

Define a relation $\sim$ on $\mathbb{R}^2$ by stating that $(a, b) \sim (c, d)$ if and only if $a^2 + b^2 \leq c^2 + d^2$. Show that $\sim$ is reflexive and transitive by not symmetric.
\hr

Reflexivity: $a^2 + b^2 = a^2 + b^2 \implies a^2 + b^2 \leq a^2 + b^2 \implies (a, b) \sim (a, b)$.

\smallskip

Transitivity: If $a^2 + b^2 \leq c^2 + d^2$ and $c^2 + d^2 \leq e^2 + f^2$, then $a^2 + b^2 \leq e^2 + f^2$, and so $(a, b) \sim (c, d)$ and $(c, d) \sim (e, f)$ implies $(a, b) \sim (e, f)$.

\smallskip

Symmetry: For any $(a, b), (c, d) \in \mathbb{r}^2$ such that $a^2 + b^2 < c^2 + d^2$, then $(a, b) \sim (c, d)$, but $(c, d) \nsim (a, b)$.

\section{}\label{sec:27}

Show that an $m \times n$ matrix gives rise to a well-defined map from $\mathbb{R}^n$ to $\mathbb{R}^m$.
\hr

For any arbitrary matrix $A \in \mathbb{R}^m \times \mathbb{R}^m$, define the function $f_A: \mathbb{R}^n \rightarrow \mathbb{R}^m$ such that $f_A(\vec{v}) = A\vec{v}$. As this multiplication can clearly be performed for every vector $\vec{v} \in \mathbb{R}^n$, giving a single unique vector in $\mathbb{R}^m$, we have that $f_A$ is a well-defined map for all $A$.
\pagebreak
\section{}\label{sec:28}

Find the error in the following argument by providing a counterexample. ``The reflexive property is redundant in the axioms for an equivalence relation. If $x \sim y$, then $y \sim x$ by the symmetric property. Using the transitive property, we can deduce that $x \sim x$.''
\hr

Let $X$ be any set and let $R$ be the empty relation on $X$; that is, the empty subset $\emptyset \subset X \times X$. Vacuously, $R$ is both symmetric and transitive. However, for any $x \in X$, it is certainly not true that $(x, x) \in R$. Therefore, $R$ is not reflexive.

\section{}\label{sec:29}

Define a relation on $\mathbb{R}^2 \setminus \{(0, 0)\}$ by letting $(x_1, y_1) \sim (x_2, y_2)$ if there exists a nonzero real number $\lambda$ such that $(x_1, y_1) = (\lambda x_2, \lambda y_2)$. Prove that $\sim$ defines an equivalence relation on $\mathbb{R}^2 \setminus \{(0, 0)\}$. What are the corresponding equivalence classes?
\hr

Reflexivity: For any $(x, y) \in \mathbb{R}^2 \setminus \{(0, 0)\}$, this is true for $\lambda = 1$.

\smallskip

Symmetry: For any $(x_1, y_1), (x_2, y_2) \in \mathbb{R}^2 \setminus \{(0, 0)\}$ such that $(x_1, y_1) \sim (x_2, y_2)$, or equivalently $(x_1, y_1) = (\lambda x_2, \lambda y_2)$, we thus have that $(x_2, y_2) = \left(\frac{1}{\lambda}x_1, \frac{1}{\lambda}y_1\right)$, and so $(x_2, y_2) \sim (x_1, y_1)$.

\smallskip

Transitivity: For any three elements $(x_1, y_1), (x_2, y_2), (x_3, y_3) \in \mathbb{R}^2$ such that $(x_1, y_1) \sim (x_2, y_2)$ and $(x_2, y_2) \sim (x_3, y_3)$, we thus have that $(x_1, y_1) = (\lambda x_2, \lambda y_2)$ and $(x_2, y_2) = (\mu x_3, \mu y_3)$. Therefore, $(x_1, y_1) = (\lambda\mu x_3, \lambda\mu y_3)$, and so $(x_1, y_1) \sim (x_3, y_3)$.

\smallskip

Each of these equivalence classes defines a line on the plane that passes through the origin (modulo $(0,0)$).